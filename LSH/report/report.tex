\documentclass[a4paper, 11pt]{article}
\usepackage{graphicx}
\usepackage{amsmath}
\usepackage{algorithm2e}
\usepackage[pdftex]{hyperref}
\usepackage{listings}
\usepackage{xcolor}

\lstset{ %
  language=Python,
  basicstyle=\ttfamily,
  otherkeywords={self},             
  keywordstyle=\ttfamily\color{blue!90!black},
  keywords=[2]{True,False, np},
  keywords=[3]{ttk},
  keywordstyle={[2]\ttfamily\color{blue!80!orange}},
  keywordstyle={[3]\ttfamily\color{red!80!orange}},
  emph={MyClass,__init__},          
  emphstyle=\ttfamily\color{red!80!black},    
  stringstyle=\color{green!80!black},
  showstringspaces=false      
}

% Lengths and indenting
\setlength{\textwidth}{16.5cm}
\setlength{\marginparwidth}{1.5cm}
\setlength{\parindent}{0cm}
\setlength{\parskip}{0.15cm}
\setlength{\textheight}{22cm}
\setlength{\oddsidemargin}{0cm}
\setlength{\evensidemargin}{\oddsidemargin}
\setlength{\topmargin}{0cm}
\setlength{\headheight}{0cm}
\setlength{\headsep}{0cm}

\renewcommand{\familydefault}{\sfdefault}

\title{Data Mining: Learning from Large Data Sets - Fall Semester 2015}
\author{member1@student.ethz.ch\\ member2@student.ethz.ch\\ member3@student.ethz.ch\\}
\date{\today}

\begin{document}
\maketitle

\section*{Approximate near-duplicate search using Locality Sensitive Hashing} 
In this project we applied Locality Sensitive Hashing (LSH) to select pair of near-duplicates from a set of videos. The input consists of a long list of lines, where every line contains the video ID and a set of shingles for that video. 
\section{Mapper}
In order to implement LSH we need to compute a signature for each video. The calculation of the signature was made using a set of 100 hash functions. Every hash function is of the form:
\begin{equation}
h_i(r) = a_ir + b_i
\end{equation}
where $a_i$ and $b_i$ are random numbers. Therefore before starting to read the input we computed these two random values for every hash function, i.e. a random matrix of size $100\times 2$.
\begin{lstlisting}
hash_functions = np.random.randint(MAX_INT, size=(HASH_FUNC_NUM, 2))
\end{lstlisting}
A signature for video $v$ is then computed using the following algorithm:
\begin{algorithm}
\For{i in 1:100}{
	signature[i] $\leftarrow \infty$
}
\For{r in shingle}{
	\For{j in 1:100}{
		hash $\leftarrow$ h(r) \% MAX\_INT
		signature[i] $\leftarrow$ min(signature[i], hash)
	}
}
\end{algorithm}

MAX\_INT is the maximum 16 bits integer. Each signature will be split into 10 segments; therefore every segment is a vector of length 10. Every segment will be then mapped into a bucket; so the number of buckets is also 10. Hence before reading the input we need to initialize the hash function that maps a segment into a bucket. These hash functions have the following form:
\begin{equation}
h(\textbf{s}) = \sum_{i = 1}^{10} c_is_i + b_i
\end{equation}
The hash functions are initialized using the following command:
\begin{lstlisting}
bucket_hash_functions = np.random.randint(MAX_INT, size=(BUCKET_SIZE + 1)).
\end{lstlisting}

\end{document} 
